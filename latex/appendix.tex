% Not sure where this content goes yet... but here's an explaination....
One of our contributions is a model that estimates the page load time for a given cache hit ratio. This model is useful for those determining how many resources (money, effort, time, etc) to put into developing an effective cache (such as a CDN). 
That is, with our model, one can estimate the return on investment that is gained by caching.

We develop a continuous probability model (i.e., we assume that the mass of all the resource objects on the page sums to 1).

% This should go in the background section. Make sure none of the variables conflict with anything else in the paper.


We categorize all objects on the critical path into two distinct groups:

Fraction of all objects on the critical path that are cached (these will incur 0 network delay).
    
\begin{align*}
P(X|K) \\
= P(X) && \text{[since X and K are independent]} \\
= X
\end{align*}

Fraction of all items on the critical path that are not cached (these will incur their original network delay):

\begin{align*}
P(1-X | K) \\
= P(1-X) && \text{[since X and P are independent]} \\
= 1-X
\end{align*}

At a given cache hit ratio X, we expect the total network delay on the critical path to be (assuming that a cached item incurs zero network delay):
\begin{align*}
E_n[X] = [X * N * 0] + [(1 - X) * N * 1] \\
= (1 - X) * N
\end{align*}

Thus, the total expected PLT given a hit ratio X is:
\begin{align*}
E_plt[X] = C + E_n[X] - f(C,N,X) \\
= C + [(1 - X) * N] - f(C,N,X)
\end{align*}

Where f(C,N,X) is the overlap between computation time and network time on the critical path.