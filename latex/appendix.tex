% Not sure where this content goes yet... but here's an explanation....
Here we derive the page load time performance model stated in~~\S\ref{subsec:model}. We
develop a continuous probability model (i.e., we assume that the size of all
objects sums to 1.0), since we are not aiming for a high degree of accuracy,
and since discrete models are more unwieldly.

%This model is useful for those determining how many resources (money, effort, time, etc) to put into developing an effective cache (such as a CDN).
%That is, with our model, one can estimate the return on investment that is gained by caching.

We introduce one term in addition to those in~~\S\ref{subsec:model}:

\noindent--Let $K$ denote the fraction of objects that are on the critical path.

In our derivation we override the terms $X$ and $K$ to also denote random
variables: the event that a given object is cached, and the event that a given
object is on the critical path.

We make two simplifying assumptions in deriving our model. First, we assume that the probability that an
object is present in the cache is independent of the probability that the object is on the critical path.
Second, we assume that the critical path does not change as we vary the cache hit
ratio. That is, we treat $K$ as a constant.

We categorize all objects on the critical path into two distinct groups.
The first category is the fraction of all objects on the critical path that are cached (these
will incur 0 network delay, assuming that the cache is colocated with the
browser):

\begin{align*}
Pr(X|K) & =  Pr(X) & \text{[since $X$ and $K$ are independent]} \\
& = X &
\end{align*}

The second category is the fraction of all items on the critical path that are not cached (these will incur their original network delay):

\begin{align*}
Pr(1-X | K) & = Pr(1-X) & \text{[since $X$ and $K$ are independent]} \\
& = 1-X &
\end{align*}

At a given cache hit ratio $X$, the expected value of the total network delay
on the critical path is (assuming that a cached item incurs zero network delay):
\begin{align*}
E_N[X] & = Pr(X|K) \cdot 0 + Pr(1-X|K) \cdot N \\
& = (1 - X) \cdot N
\end{align*}

Thus, the total expected PLT given a hit ratio $X$ is:
\begin{align*}
E_{PLT}[X] & = C + E_N[X] - f(C,N,X) \\
& = C + (1 - X) \cdot N - f(C,N,X)
\end{align*}

% TODO(cs): To generalize this model to a case where the cache is located far away from
% the browser... we could introduce a non-zero fetch cost associated with
% cached items.
