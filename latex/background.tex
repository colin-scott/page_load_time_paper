It is important to understand how browsers load web pages. When a user directs the browser to a new URL, the browser's Object Loader fetches the root HTML object, as depicted
in Figure~\ref{fig:network-diagram}. The HTML Parser launches additional
fetch requests for each linked resource within the HTML. In this way, the browser incrementally generates the DOM.
As the page loads, the Rendering component paints the UI.

From the user's perspective, the performance of a website can be defined according to a number of different metrics~\cite{above-the-fold,speed-index}. Here,
we focus on page load time, which is easy to measure
and loosely standardized across browsers~\cite{w3c-onload}.

\textbf{Critical Path}. Critical path analysis is a method for analyzing the performance of parallel processes. Web pages are comprised of many objects, such as images, Javascript, CSS, and HTML.
Each of these objects is handled by multiple browser tasks: it must be
fetched, parsed or evaluated, and rendered. We refer to the non-overlapping
delays involved in parsing and rendering an object as its `computational
delay,' and refer to the fetch delay as its `network delay.'

Certain tasks are dependent on others and must wait until their
predecessor tasks have completed. The critical path of a web page is the longest chain of dependent browser tasks such that reducing the length of any task not on the critical
path will not change the critical path (See Figure~\ref{fig:plt-diagram})~\cite{sarkar1987partitioning}.
%\colin{Should make a clearer definition for `tasks' and `objects'. The web page is composed of multiple objects, each of which must be fetched (1 task), parsed or evaluated (1 task), and rendered (1 task).}

% TODO(cs): say whether it's possible that objects on the critical path can
% overlap or not. I think it is possible, e.g., if HTML is partially loaded,
% the browser can kick of other (dependant) tasks before the HTML has completed
% loading.

\textbf{Page Load Time}. Roughly speaking, page load time (PLT) is the elapsed time from the moment a user requests a web page to the moment all resources on the page have been loaded~\cite{page-speed}.
We measure PLT by listening to the browser's Javascript \texttt{onload} event,
which fires in most browsers when all resources have been added to the DOM, and all images,
scripts, links, and sub-frames have finished loading~\cite{w3c-onload}.

PLT is determined by the length of the page's critical path. In
Figure~\ref{fig:plt-diagram}, the network and computational delay for the HTML, CSS, JS, and JPEG
objects determine the PLT. If we were to decrease the delay for loading the
PNG object, the critical path would remain the same, and therefore, the PLT would not change.

%\colin{Need to make clear that certain objects (tasks) are dependent on others. When a task is dependent on another, it must wait until its dependencies (tasks) have completed. E.g., all tasks are dependent on the HTML parsing task for the root HTML)}
%\jamshed{Added this to "Critical Path" section

\subsection{Performance Model}
\label{subsec:model}

We develop an intuition for how caching should affect page load
time and develop a simple performance model. We show the details of how we derive
this model in the \S\ref{sec:appendix}.

First, consider the following terms:

\noindent-- Let $X$ denote a given cache hit ratio. We define a cache hit ratio as the fraction
of all objects in a web page that are served by a cache. Note that the maximum
value of $X$ is the fraction of cacheable items on the page, which may be less than 1.

\noindent-- Let $N$ denote the summation of network fetch delays for all objects on the
critical path for a cold ($X=0$) page load.
% Why do we need to assume the hit ratio is 0 when we just said K is independent of the hit ratio above?
% Colin: N and C are constants, i.e. their value needs to be set (measured)
% with a fixed value of X. It's convenient and simplest to set the fixed value
% to X to 0, i.e. ensure that the initial page load did not have any cached
% hits. We could set the initial value of X to something else, but 0 is simplest.

\noindent-- Let $C$ denote the summation of computational delays for all
objects on the critical path for a cold ($X=0$) page load.

\noindent-- Let $f(C,N,X)$ denote the overlap between computational delay and
network delay on the critical path. For most purposes, we can treat $f(C,N,X)$
as being close to 0, since dependent objects by definition cannot be fetched
until their predecessors are at least partially fetched and evaluated.

We can then model the expected value of the PLT at a given cache hit ratio as:
\begin{align*}
E_{PLT}[X] = C + (1 - X) \cdot N - f(C,N,X)
\end{align*}

% TODO(cs): show off some predictions we can make with this model.
% TODO(cs): put Figure 13 from WProf here.
