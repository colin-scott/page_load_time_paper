% Not sure where this content goes yet... but here's an explaination....
One of our contributions is a model that estimates the page load time for a given cache hit ratio. This model is useful for those determining how many resources (money, effort, time, etc) to put into developing an effective cache (such as a CDN). 
That is, with our model, one can estimate the return on investment that is gained by caching.

We develop a continuous probability model (i.e., we assume that the mass of all the resource objects on the page sums to 1).

% This should go in the background section. Make sure none of the variables conflice with anything else in the paper.

Terms:
\begin{itemize}
\item X = The cache hit ratio. This is defined as the fraction of all objects in a web page that are present in the cache.
    \begin{itemize}
    \item Assumption: The probability that an object is actually cached is independent of the probability that the object is on the critical path.
    \item Note: the max value of X is the fraction of cacheable items, which may be less than 1. 
    \end{itemize}
\item K = Fraction of all objects on the critical path
    \begin{itemize}
    \item Assumption: The critical path does not change as we vary the cache hit ratio. That is, we treat K as a constant.
    \end{itemize}
\item N = The summation of network delays for all objects on the original (cache hit ratio = 0) critical path.
% Why do we need to assume the hit ratio is 0 when we just said K is independent of the hit ratio above?
\item C = The summation of computational delays for all objects on the original (cache hit ratio = 0) critical path.
\end{itemize}

We categorize all objects on the critical path into two distinct groups:

Fraction of all objects on the critical path that are cached (these will incur 0 network delay).
    
\begin{align*}
P(X|K) \\
= P(X) && \text{[since X and K are independent]} \\
= X
\end{align*}

Fraction of all items on the critical path that are not cached (these will incur their original network delay):

\begin{align*}
P(1-X | K) \\
= P(1-X) && \text{[since X and P are independent]} \\
= 1-X
\end{align*}

At a given cache hit ratio X, we expect the total network delay on the critical path to be (assuming that a cached item incurs zero network delay):
\begin{align*}
E_n[X] = [X * N * 0] + [(1 - X) * N * 1] \\
= (1 - X) * N
\end{align*}

Thus, the total expected PLT given a hit ratio X is:
\begin{align*}
E_plt[X] = C + E_n[X] - f(C,N,X) \\
= C + [(1 - X) * N] - f(C,N,X)
\end{align*}

Where f(C,N,X) is the overlap between computation time and network time on the critical path.