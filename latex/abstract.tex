A recent NSDI paper~\cite{flywheel} reported that
increasing the cache hit ratio for an HTTP proxy from 22\% to 32\% improved
median page load time (PLT) for mobile clients by less than 2\%.
We argue that there are two main causes for this weak improvement:
objects on the critical path are often not cached and the limited computational power of mobile devices causes computational delays to  comprise a large portion of the critical path. %and this effect is amplified by a low rate of cache hits for objects on the critical path.

% for desktop:
%  C = 0.15
%  N = 0.85

% for mobile:
%  C = 0.4
%  N = 0.6

% C:N -> 1

Both of
these factors were, in fact, outlined by a previous analysis of desktop
web performance~\cite{wang2013demystifying}. However, we (as the authors of the HTTP proxy~\cite{flywheel}) did not properly understand the analysis, and could have saved ourselves substantial engineering costs if we had. We therefore argue for the need to highlight this prior
analysis, and extend the analysis
to include mobile devices with slow CPUs, precise cache hit ratios,
and a controlled reproduction of the
HTTP proxy caching
results~\cite{flywheel}. %, and observe comparable PLT reductions of 1\% in the median case.
In the extreme case of a perfect cache hit ratio, desktop page load times are
improved notably by 34\% compared to no caching, but mobile page
load times only improve by 13\% in the median case.
We extract a back-of-envelope performance model from these results to
help understand their underlying causes.

% which
% assesses how different network and desktop processing speeds
% affect the PLT---processors of mobile devices are analogous to
% desktop processors that are slower. <rest of abstract here>

% OLD:
% A recent NSDI paper~\cite{flywheel} reported a counterintuitive result:
% increasing the cache hit ratio of their HTTP proxy from 22\% to 32\% only improved median page load time for mobile clients by 1-2\%.
% %In their NSDI paper~\cite{flywheel}, Google reported a counterintuitive result: increasing the cache hit ratio of their Flywheel HTTP proxy from 22\% to 32\% only improved median page load time for mobile clients by 1-2\%.
% Here, we seek to understand why improved caching had such a negligible effect on mobile performance.
% We reproduced these results in a controlled, mobile environment and observed
% comparable reductions of 1\% in the median case.
% Further, we show that with a perfect cache hit ratio, desktop page load times are improved notably by 34\% compared to no caching, but mobile page load times only improve by 13\% in the median case. Our contribution includes a model for predicting the estimated page load time for a given cache hit ratio.
% We further demonstrate that CPU speed is the key bottleneck preventing mobile devices from benefiting from web caching.

%\colin{when we get the partial caching results, you might want to replace the 35\%, 13\% statistics with the partial caching reuslts.}

% CS: removing this for now, since it isn't required in the CFP:
%    https://www.usenix.org/conference/atc16/requirements-authors
%\\\\
%\noindent
%\textbf{Keywords:} mobile, web performance, web caching
