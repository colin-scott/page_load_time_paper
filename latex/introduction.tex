\label{intro}
%With the increase in mobile Internet traffic, there is a larger emphasis on methods of improving web performance for mobile devices.
Web caching is widely used to reduce network link utilization, decrease server load and data usage, improve reliability for origin web servers, and improve latency for end hosts. 
Here, we focus exclusively on web caching's effect on latency, as measured by web page load time.

Flywheel~\cite{flywheel}, Google's HTTP proxy for mobile devices, increased
its overall cache hit ratio from 22\% to 32\%, yet observed only a 1-2\% reduction in page load time in the median case.
Here we seek to gain a deeper understanding of why improved caching had such a negligible effect on performance.

Our contributions are as follows.
We present a methodology
% Commenting this out because they don't know what WPR and Telemetry are yet.
%based on Web Page Replay~\cite{wpr} and Telemetry~\cite{telemetry}
for evaluating the performance impact of web caching for hypothetical levels of cache hit ratios, in a controlled environment. We use this methodology to compare the effects of partial (hit ratio of 20\% to 30\%) and 
perfect caching (hit ratio of 100\%) versus no caching for a set of 400 Alexa web pages~\cite{alexa} on both a mobile device and a desktop browser.

We demonstrate that by increasing the cache hit ratio from 20\% to 30\%, there is only a meager 1\% reduction in mobile web page load time.
Further, we show that with perfect caching, desktop page load times are improved notably by 34\% compared to no caching, but mobile page load times only improve by 13\% in the median case.
We further show that CPU speed is the key resource bottleneck preventing mobile devices from benefiting significantly from web caching.

Our results suggest that the common wisdom that web caching is a crucial part of web performance optimization may not hold for mobile clients. Content providers may want to reconsider where they focus their efforts, especially as the volume of mobile traffic begins to overtake desktop traffic.