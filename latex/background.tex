Before describing our experimental apparatus, it is important to understand how browsers load web pages. When a user directs the browser to a new URL, the browser's Object Loader fetches the root HTML object, as depicted in Figure~\ref{fig:network-diagram}. The HTML Parser then launches additional fetch requests for each linked resource while it generates the DOM tree. As the page loads, the Rendering component paints the browser’s UI.

From the user's perspective, the performance of a website can be defined according to a number of different metrics~\cite{above-the-fold,speed-index}. Here, we focus on page load time, which is easy to measure
 and loosely standardized across browsers~\cite{w3c-onload}.

\textbf{Critical Path}. Critical path analysis is a method for analyzing the performance of parallel processes. Web pages are comprised of many objects, such as images, Javascript, CSS, and HTML. Each of these objects is handled by multiple browser tasks: it must be fetched, parsed or evaluated, and rendered. Certain tasks are dependent on others, and must wait until their predecessor tasks have completed. The critical path of a web page is the longest chain of dependent browser tasks such that reducing the length of any task not on the critical path will not change the critical path (See Figure~\ref{fig:plt-diagram})~\cite{sarkar1987partitioning}.
%\colin{Should make a clearer definition for `tasks' and `objects'. The web page is composed of multiple objects, each of which must be fetched (1 task), parsed or evaluated (1 task), and rendered (1 task).}

\textbf{Page Load Time}. Loosely speaking, page load time (PLT) is the elapsed time from the moment a user requests a web page to the moment all resources on the page have been loaded~\cite{page-speed}. PLT is determined by the length of the page's critical path.
For example, in Figure~\ref{fig:plt-diagram}, decreasing the time to load the PNG object would not change the critical path and thus would not change the PLT.
In the browser, PLT is calculated by listening to Javascript's  \texttt{onload} event to determine when all resources (including Javascript) have been rendered~\cite{w3c-onload}.
%\colin{Need to make clear that certain objects (tasks) are dependent on others. When a task is dependent on another, it must wait until its dependencies (tasks) have completed. E.g., all tasks are dependent on the HTML parsing task for the root HTML)}
%\jamshed{Added this to "Critical Path" section