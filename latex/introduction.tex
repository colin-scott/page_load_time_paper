\label{intro}
%With the increase in mobile Internet traffic, there is a larger emphasis on methods of improving web performance for mobile devices.
Web caching is widely used to reduce network link utilization, decrease server load and data usage, improve reliability for origin web servers, and improve latency for end hosts.
Here, we focus exclusively on web caching's effect on latency, as measured by web page load time.

Flywheel~\cite{flywheel}, Google's HTTP proxy for mobile devices, increased
its overall cache hit ratio from 22\% to 32\%, yet observed only a 1--2\% reduction in page load time in the median case.
As the designers of Flywheel, we were initially surprised by this weak
improvement. If we had been able to predict that caching would have such
negligible effects, we could have saved ourselves substantial engineering costs. %Here we seek to gain a deeper understanding of these results.

In Section 5 of their paper on measuring the critical paths of web page 
loads~\cite{wang2013demystifying}, Wang et al. seek to demonstrate use 
cases for their measurement tool. Two of these use cases---an analysis of varying CPU speeds, and an analysis of varying cache occupancy levels---in fact outline the likely root causes for Flywheel's result.

In this paper we seek to highlight Wang et al.'s analysis, as we suspect that
we are not alone in our prior misconception that caching should
improve latency. We also extend
their analysis along several dimensions. We present 
a methodology for varying cache hit ratios at fine granularity,
and measuring caching's effect on page load times in a controlled and 
reproducible manner.
We use this methodology to measure caching's performance effects on both a mobile device and a desktop browser.
In our controlled environment, we reproduce Flywheel's reported cache hit
ratio increase for
a set of
400 Alexa web pages~\cite{alexa}. We find a comparable 1\% decrease in PLT in the median
case.
%Consistent with Flywheel's reported result, we demonstrate that by increasing the cache hit ratio from 20\% to 30\%, there is only a meager 1\% reduction in mobile web page load time.
In the extreme case of a perfect cache hit ratio, we find that desktop page load times are improved notably by 34\% compared to no caching, but mobile page load times only improve by 13\% in the median case.

We develop a back-of-the-envelope model, and calibrate its parameters to
our empirical results, to help understand the underlying causes.
Our model indicates that CPU speed is the key resource bottleneck preventing mobile devices from benefiting significantly from web caching. 
The analysis by Wang et al. seems to further indicate that objects on the critical path are often not in cache (possibly because they are not cacheable).

%These two findings are also consistent with previous desktop performance analysis~\cite{wang2013demystifying}.

The key implication of these results is that caching's favorable improvements on desktop page load times do not carry over well to mobile clients. Content providers may want to think twice about expending resources on caching as a means for improving latency, especially as the volume of mobile traffic begins to overtake desktop traffic.

% TODO(cs): attempt to argue that others, besides the Flywheel authors, also
% suffer from the same misconception about web caching's potential benefits?
